%%%%%%%%%%%%%%%%%%%%%%%%%%%%%%%%%%%%%%%%%%%%%%%%%
% Formatowanie list wyliczeniowych, wypunktowań i własnych otoczeń
%%%%%%%%%%%%%%%%%%%%%%%%%%%%%%%%%%%%%%%%%%%%%%%%%
\usepackage{enumitem} % pakiet pozwalający zarządzać formatowaniem list wyliczeniowych
\setlist{noitemsep,topsep=4pt,parsep=2pt,partopsep=4pt,leftmargin=*}
\setenumerate{labelindent=0pt,itemindent=0pt,leftmargin=!,label=\arabic*.} % \arabic, \alph
\setlistdepth{4}
\setlist[itemize]{leftmargin=1.5cm}  % margines listy
\setlist[itemize,1]{label=$\bullet$}
\setlist[itemize,2]{label=\normalfont\bfseries\textendash}
\setlist[itemize,3]{label=$\ast$}
\setlist[itemize,4]{label=$\cdot$}
\renewlist{itemize}{itemize}{4}

\makeatletter
\renewenvironment{quote}{
    \begin{list}{}
        {
            \setlength{\leftmargin}{1em}
            \setlength{\topsep}{0pt}%
            \setlength{\partopsep}{0pt}%
            \setlength{\parskip}{0pt}%
            \setlength{\parsep}{0pt}%
            \setlength{\itemsep}{0pt}
        }
        }{
    \end{list}}
\makeatother

%%%%%%%%%%%%%%%%%%%%%%%%%%%%%%%%%%%%%%%%%%%%%%%%%
% Kropki po numerach sekcji
%%%%%%%%%%%%%%%%%%%%%%%%%%%%%%%%%%%%%%%%%%%%%%%%%
\makeatletter
\def\@seccntformat#1{\csname the#1\endcsname.\quad}
\def\numberline#1{\hb@xt@\@tempdima{#1\if&#1&\else.\fi\hfil}}
\makeatother

%%%%%%%%%%%%%%%%%%%%%%%%%%%%%%%%%%%%%%%%%%%%%%%%%
% Pakiet do generowania indeksu
%%%%%%%%%%%%%%%%%%%%%%%%%%%%%%%%%%%%%%%%%%%%%%%%%
\DisemulatePackage{imakeidx}
\usepackage[makeindex,noautomatic]{imakeidx}
\makeindex

\makeatletter
\renewenvironment{theindex}
{\vskip 10pt\@makeschapterhead{\indexname}\vskip -3pt%
    \@mkboth{\MakeUppercase\indexname}%
    {\MakeUppercase\indexname}%
    \vspace{-3.2mm}\parindent\z@%
    \renewcommand\subitem{\par\hangindent 16\p@ \hspace*{0\p@}}%%
    \phantomsection%
    \begin{multicols}{2}
        \thispagestyle{plain}
        \parindent\z@
        %\parskip\z@ \@plus .3\p@\relax
        \let\item\@idxitem}
        {\end{multicols}\clearpage}
\makeatother

%%%%%%%%%%%%%%%%%%%%%%%%%%%%%%%%%%%%%%%%%%%%%%%%%
% Spisy
%%%%%%%%%%%%%%%%%%%%%%%%%%%%%%%%%%%%%%%%%%%%%%%%%
\newcommand{\bibliografia}[1]{
    \chapterstyle{noNumbered}
    \bibliographystyle{plabbrv} % plalpha, plabbrv
    \bibliography{#1}
    \appendix
}

\newcommand{\spistresci}{
    \chapterstyle{noNumbered}
    \mbox{}\pdfbookmark[0]{\contentsname}{content}
    \label{spis-tresci}
    \tableofcontents*
}

\newcommand{\spisrysunkow}{
    \newpage
    \chapterstyle{noNumbered}
    % \mbox{}\pdfbookmark[0]{\listfigurename}{listfigurename}
    \addcontentsline{toc}{chapter}{\listfigurename}
    \listoffigures*
}

\newcommand{\spistabel}{
    \newpage
    \chapterstyle{noNumbered}
    % \mbox{}\pdfbookmark[0]{\listtablename}{listtablename}
    \addcontentsline{toc}{chapter}{\listtablename}
    \listoftables*
}

\newcommand{\indeksrzeczowy}{
    \chapterstyle{noNumbered}
    % \mbox{}\pdfbookmark[0]{\indexname}{indexname}
    \addcontentsline{toc}{chapter}{\indexname}
    \printindex
    \newpage
}
