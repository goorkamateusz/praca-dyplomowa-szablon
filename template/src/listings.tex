\usepackage{listings} % pakiet do prezentacji kodu.
% Aby rozwiązać problem polskich znaków:
\lstset{literate=%-
	{ą}{{\k{a}}}1 {ć}{{\'c}}1 {ę}{{\k{e}}}1 {ł}{{\l{}}}1 {ń}{{\'n}}1 {ó}{{\'o}}1 {ś}{{\'s}}1 {ż}
	{{\.z}}1 {ź}{{\'z}}1 {Ą}{{\k{A}}}1 {Ć}{{\'C}}1 {Ę}{{\k{E}}}1 {Ł}{{\L{}}}1 {Ń}
	{{\'N}}1 {Ó}{{\'O}}1 {Ś}{{\'S}}1 {Ż}{{\.Z}}1 {Ź}{{\'Z}}1 }

\lstloadlanguages{csh,CIL,SQL,XML,sh,HTML} % załadowane języki

\usepackage{color}
\definecolor{red}{rgb}{0.6,0,0}
\definecolor{blue}{rgb}{0,0,0.6}
\definecolor{green}{rgb}{0,0.8,0}
\definecolor{cyan}{rgb}{0.0,0.6,0.6}
\definecolor{cloudwhite}{rgb}{1,1,1}

\lstset{ % domyślna deklaracja
	basicstyle=\linespread{1}\footnotesize\ttfamily,
	numbers=left,
	numberstyle=\tiny,
	numbersep=5pt,
	tabsize=2,
	extendedchars=true,
	breaklines=true,
	breakautoindent=true,
	% postbreak=\mbox{\textcolor{red}},
	frame=b,
	stringstyle=\color{blue}\ttfamily,
	showspaces=false,
	showtabs=false,
	xleftmargin=17pt,
	xrightmargin=0pt,
	framexleftmargin=17pt,
	framexrightmargin=0pt,
	framexbottommargin=4pt,
	commentstyle=\color{green},
	morecomment=[l]{//}, %use comment-line-style!
	morecomment=[s]{/*}{*/}, %for multiline comments
	showstringspaces=false,
	keywordstyle=\color{cyan},
	identifierstyle=\color{red},
	% backgroundcolor=\color{cloudwhite}
}

% \usepackage{caption}
% \DeclareCaptionFormat{listing}{\colorbox{cloudwhite}{\parbox{\textwidth}{\hspace{15pt}#1#2#3}}}
% \captionsetup[lstlisting]{format=listing, singlelinecheck=false, margin=0pt, font={bf,footnotesize}}
